% ! TEX root = ../mechanics.tex 


\part{理论力学}

\chapter{静力学公理和受力分析}

\section{公理}
静力学一共由五条公理.
\begin{enumerate}
\item 力的平行四边形法则 \\
作用在物体上同一点的两个力,可以合成为一个合力,合力的作用点也在该点,
合力的大小和方向由这两个力为边构成的平行四边形法则的对角线确定,
\marginpar{制作插图}也就是合力矢等于这两个力矢的矢量和.
\item 二力平衡条件 \\
作用在同一刚体上的两个力,使刚体保持平衡的必要条件和充分条件是:这两个
力大小相等,方向相反,作用在同一直线上.三个条件缺一不可.
\item 加减平衡力系原理 \\
在任一原有力系上加上或减去任意的平衡力系,与原力系对刚体的作用效果
等效.由此可以得到两个推论.
\begin{enumerate}
\item 力的可传性 \\
作用在刚体上某点的力,可以沿着它的作用线移到刚体内的任意一点,并不改变
该力对刚体的作用.
\item 三力平衡汇交原理 \\
刚体在三个力的作用下平衡,若其中两个力的作用线交于一点,则第三个力的作用
线必通过此汇交点,且三个力位于同一个平面内.
\end{enumerate}
\item 作用和反作用定律 \\
作用力和反作用力总是同时存在, 两个力的大小总是大小相等,方向相反,沿着
同一条直线,分别作用在两个相互作用的物体上.
\begin{notice}
作用力和反作用力与二力平衡的区别在于,作用力和反作用
力作用在两个不同的物体上,而一对平衡力作用在同一个物体上.
\end{notice}
\item 刚化原理 \\
变形体在某一力系作用下处于平衡,如此将变形体刚化为刚体,其平衡状态保持不变.
\end{enumerate}

物体
