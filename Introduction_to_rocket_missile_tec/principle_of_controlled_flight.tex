% ! TEX root = ../mechanics.tex

\chapter{导弹的控制飞行原理}
导弹和普通武器的{\color{blue}本质区别}在
于导弹有制导系统.{\color{blue}制导系统}的基本任务
是确定导弹与目标的相对位置,操纵导弹飞行,
在一定的准确度下,引导导弹沿预定的弹道
飞向目标.

{\bfseries 控制力}\index{控制力}是控制
导弹质心运动的力.{\bfseries 操纵力}\index{操纵力}是
操纵导弹绕质心发生
转动的力.
\begin{notice}
	产生和改变控制力的方法:
	\begin{equation*}
		\color{titleblue}
		\begin{cases}
			 \text{有翼导弹:主要靠空气动力产生
			控制力}                     \\
			 \text{无翼导弹:主要靠发动机推力产生
				控制力}
		\end{cases}
	\end{equation*}
\end{notice}
用来产生操纵力矩的元件叫做{\color{blue}操纵元件}.
\begin{note}
	操纵元件除了产生操纵力矩对导弹起操纵作用
	外,它还可以对导弹起飞行稳定的作用.
\end{note}

\section{产生和改变控制力的方法}
\begin{equation*}
	\color{titleblue}
	\begin{cases}
		 利用空气动力来产生和改变控制力
		\begin{cases}
			轴对称导弹: & 有两对弹翼,在纵向对称平面           \\
			       & 和侧向对称平面内
			都能产生较             \\
			       & 大的空气动力.  \\
			面对面导弹: & 弹翼能产生
			较大的气动力,           \\
			       & 弹身和尾翼的空气
			动力较小.
		\end{cases}   \\
		 利用发动机推力来产生和改变控制力
	\end{cases}
\end{equation*}
\begin{note}
	这两种产生和改变控制力的方法的共同特点是:
	都需要依靠操纵元件使导弹绕质心转动.
\end{note}
\begin{notice}
	直接产生法向控制力的方法
	\begin{equation*}
		\color{titleblue}
		由火箭发动机直接产生法向控制力
		\begin{cases}
			 依靠旋转弯管型喷管直接产生法向
			控制力                \\
			 依靠侧向喷管直接产生法向控制力
		\end{cases}
	\end{equation*}
\end{notice}
\section{导弹的操纵元件}
\begin{equation*}
	\color{titleblue}
	\begin{cases}
		 空气动力舵面
		\begin{cases}
			 全动舵            \\
			 位于弹翼或尾翼后缘的舵和副翼 \\
			 翼尖舵和翼尖副舵       \\
			 转子副翼           \\
			 旋转弹翼           \\
			 扰流片
		\end{cases} \\
		 燃气动力操纵元件
		\begin{cases}
			 燃气舵               \\
			 摆动发动机             \\
			 摆动喷管              \\
			 摆帽(喷气流偏转器)        \\
			 燃气挡片              \\
			 向喷管内喷射气流或液体(二次喷射) \\
			 旋转弯管型喷管和侧喷管辅助发动机
		\end{cases}
	\end{cases}
\end{equation*}
\section{导弹制导原理}
 {\color{blue}制导系统}是完全操纵导弹飞向目标
任务所有设备的总和.
从准备发射到摧毁目标经过的三个阶
段:{\color{blue}发射控制,飞行控制,爆炸控制}.
\begin{note}
	控制系统的任务:操纵导弹执行引导系统发出的控制
	指令,控制导弹飞行目标.保证导弹在每一飞行段的
	稳定性.
\end{note}
\begin{align*}
	\color{titleblue}
	导弹的控制通道
	\begin{cases}
		 三通道控制
		\begin{cases}
			 俯仰 \\
			 偏航 \\
			 倾斜
		\end{cases} \\
		 双通道控制
		\begin{cases}
			 俯仰 \\
			 偏航
		\end{cases}
	\end{cases}
\end{align*}
\subsection{制导系统的分类}
\begin{equation*}
	\color{titleblue}
	制导系统
	\begin{cases}
		 自寻的系统
		\begin{cases}
			 主动式\text{:雷达,声学原理}         \\
			 半主动式\text{:雷达,激光}          \\
			 被动式\text{:雷达,红外,光学原理,声学原理}
		\end{cases} \\
		 遥控制导系统
		\begin{cases}
			 波束制导
       \begin{cases}
         无线电波束制导系统
         \text{:导弹自动沿雷达波束飞行}\\ 
         激光波束制导系统
         \text{:不易被干扰,精度高}
       \end{cases}\\
			 遥控指令制导\text{:目视,雷达,电视} \\
			 无线电制导                 \\
			 全球卫星制导
		\end{cases}      \\
		 自主制导
		\begin{cases}
			 惯性系统    \\
			 天文系统    \\
			 多普勒雷达系统 \\
			 地形匹配系统
		\end{cases}                  \\
		 复合制导系统
		\begin{cases}
			 自主+自动寻的系统    \\
			 自主+遥控系统      \\
			 遥控+自动寻的系统    \\
			 自主+遥控+自动寻的系统
		\end{cases}
	\end{cases}
\end{equation*}
\subsection{导弹的控制方式}
\begin{equation*}
	\color{titleblue}
	控制方式
	\begin{cases}
		 单通道控制,导弹自旋,一字舵面      \\
		 双通道控制,相互垂直的俯仰和偏航通道控制 \\
		 三通道控制,俯仰,偏航,倾斜三个通道控制
	\end{cases}
\end{equation*}
制导系统的基本要求如下
\begin{equation*}
	\color{titleblue}
	\begin{cases}
		 制导准确度(脱靶量)
		\begin{cases}
			 系统误差 \\
			 随机误差
		\end{cases}    \\
		 制导系统对目标的鉴别能力 \\
		 制导系统的可靠性     \\
		 制导系统的抗干扰能力
	\end{cases}
\end{equation*}
\begin{note}
	制导系统的制导误差主要取决于制导
	系统的动态误差,起伏误差,仪器误差.
\end{note}
\subsection{制导系统必须具备的功能}
\begin{enumerate}
	\item 导弹在飞行目标的过程中,要不断地
	      测量导弹的实际运动与理想运动之间的
	      偏差
	\item 根据偏差大小和方向形成控制指令,将
	      指令送到操纵元件,控制导弹改变运动状
	      态,消除该偏差
	\item 稳定导弹运动姿态角,使导弹始终保
	      持所需的姿态角
\end{enumerate}
\subsection{制导系统的组成}
{\bfseries 制导系统}是导引系统和控制系统的总称.
\begin{equation*}
	\color{titleblue}
	导引系统
	\begin{cases}
		测量装置 \text{:}& 测量目标和导弹的运动参数    \\
		程序装置  \text{:}&储存和发生使导弹按照预先规定
		的程序运动的参数和指令                    \\
		解算装置 \text{:}& 将测量装置测得的信息差经计算和
		变换后,形成控制指令信息输送给                \\
		     & 控制系统
	\end{cases}
\end{equation*}
\begin{equation*}
	\color{titleblue}
	控制系统
	\begin{cases}
		敏感装置 \text{:} & 感受和测量导弹的姿态角信息
		及重心运动信息                             \\
		综合装置  \text{:}& 将导引系统送来的信息与敏感装置
		送来的信息加以总和,                          \\
		      & 形成对导弹综合控制指令信息               \\
		放大变换器\text{:} & 将综合装置送来的指令信息进行
		校正,变换和功率放                           \\
		      & 大,使之称为推动执行机构工作的
		指令信息                                \\
		执行机构 \text{:} & 舵机与操纵元件组合的总称称为执行机构,
		它能根据指                               \\
		      & 令信息驱动操纵元件动作
	\end{cases}
\end{equation*}
\section{自主制导系统}
自主制导系统的基本原理
是:{\color{blue}按照发生前预先规定的程序或外界
固定的参考点作为基准来将导弹自动地导向目标}.
程序由导弹运动学参数与时间参数的一组固定关系组
成.固定的参考点可以利用卫星,星球,地理条件等.
\begin{note}
	自主自导系统的引导指令仅由弹上制导设备敏感
	地球或宇宙空间物质的物理特性产生,不与目
	标,制导站发生关联.
\end{note}
\subsection{测量,敏感装置}
\begin{equation*}
	\color{titleblue}
	\begin{cases}
		 陀螺仪
		\begin{cases}
			 定位陀螺仪,三个自由度,可测量
			两个方向的角偏移              \\
			 速率陀螺仪,两个自由度,可测量导弹
			绕某一坐标轴的角速度            \\
			 积分陀螺仪,可测量一个方向上的角偏移
		\end{cases} \\
		 飞行高度表
		\begin{cases}
			 气压式高度表,绝对高度 \\
			 无线电高度表,相对高度
		\end{cases}        \\
		 加速度计              \\
	\end{cases}
\end{equation*}
\begin{note}
	绝对高度是指到海平面的高度;相对高度是距
	地面的高度
\end{note}
陀螺仪的组成
\begin{equation*}
	\color{titleblue}
	\begin{cases}
		 转子,高速旋转,可绕轴转动    \\
		 内环架,通过轴和轴承固定在外环上 \\
		 外环架,通过轴和轴承固定在支架上
	\end{cases}
\end{equation*}
\begin{notice}
	陀螺仪的特性
	\begin{enumerate}
		\item {\bfseries 定轴性}:陀螺转子在惯性空间的
		      方位保持不变
		\item {\bfseries 进动性}:在主轴垂直的方向上
		      施加一外力矩$M$,则陀螺要绕着与外力
		      矩垂直的的方向转动
	\end{enumerate}
\end{notice}
干扰力矩的存在会引起转子轴缓慢的进动,
这个现象叫做陀螺
的 {\bfseries 漂移}\index{漂移}.
\begin{notice}
	使陀螺发生漂移的原因:
	\begin{enumerate}
		\item 轴承与传感器之间存在摩擦
		\item 陀螺仪本身制导的不对称和不平衡
	\end{enumerate}
\end{notice}
陀螺仪缓慢进动的角速度称为{\color{blue}漂移率或漂移
角速度}.
\begin{notice}
	减小漂移率的方法:
	\begin{enumerate}
		\item 增加转子的动量矩
		\item 减小干扰力
	\end{enumerate}
\end{notice}
\subsection{惯性制导系统}
{\color{blue}惯性制导}系统是利用导弹上的惯性
仪表来测量导弹的速度和坐标从而
形成指令信息来导引导弹的系统.
\begin{equation*}
  \color{titleblue}
  惯性制导系统
  \begin{cases}
    平台制导系统,将加速度计安装在
    陀螺稳定平台上\\ 
    捷联制导系统,将加速度计与导弹固连
  \end{cases}
\end{equation*}
\begin{notice}
  捷联制导系统的优缺点:
  \begin{enumerate}
    \item 优点:\\ 
      简化了系统,可靠性高
    \item 缺点:\\ 
      由于加速度计与导弹固连而
      处于相当苛刻的振动环境中,
      影响制导精度
  \end{enumerate}
\end{notice}
惯性制导系统
{\color{blue}完全自动地控制导弹飞行,具有
优秀的抗干扰能力}.但是,该系统中陀螺仪
{\color{blue}存在
漂移误差,该误差会积累}.

\subsection{天文制导系统}
利用测量恒星的方法来确定导弹的位置,一般
测量两颗恒星的位置.

天文制导系统完全自动化,不受外界的干扰.

\subsection{多普勒雷达制导系统}
多普勒制导系统往往用于复合制导系统中,用
来校正其他系统的误差.

\subsection{地形匹配系统}
利用某已知地区的地形特征作为标志,根据导弹
当下弹道测量的地形特征和预定弹道下的地形
特征做比较,来校正导弹的弹道,使导弹按照预定
路线导向目标.
\begin{note}
不能用于没有地形差别的海平面和平原地区.
\end{note}
\section{遥控制导系统}
{\color{blue}遥控制导系统}是由导弹以外的制导站
向导弹发送引导信息的制导系统.
\begin{equation*}
  \color{titleblue}
  遥控制导系统
  \begin{cases}
    指令制导系统
    \begin{cases}
      有线指令系统\\ 
      无线电指令系统
      \begin{cases}
        目视\\ 
        雷达自动跟踪
      \end{cases}\\ 
      电视指令系统
      \begin{cases}
        优点\text{:导引精度高}\\ 
        缺点\text{:容易受敌方干扰和天气影响}
      \end{cases}
    \end{cases}\\ 
    波束制导系统\\ 
    无线电导航制导系统
  \end{cases}
\end{equation*}

\begin{notice}
遥控制导系统的优缺点:
\begin{enumerate}
  \item 优点:\\ 
    制导精度高,制导距离比自寻的系统稍远,弹
    上制导设备简单
  \item 缺点:\\ 
  制导精度随导弹离制导站的距离增大而
    减小,且易受外界干扰
\end{enumerate}
\end{notice}
多用于地空,空空,空地导弹.

指令制导系统的组成
\begin{equation*}
  \color{titleblue}
  指令制导系统
  \begin{cases}
    观测装置\\ 
    指令形成装置\\ 
    控制线
  \end{cases}
\end{equation*}

电视导引系统的分类
\begin{equation*}
  \color{titleblue}
  \begin{cases}
    自动寻式的电视制导系统\\ 
    遥控式的电视指令制导系统
  \end{cases}
\end{equation*}
\section{自动寻的制导系统}
{\color{blue}自动寻的制导系统}是
目标辐射或反射的能量导引导弹去攻击
目标.
\begin{note}
自动寻的制导系统要求背景有足够的
能量对比性
\end{note}
\begin{notice}
自动寻的制导系统的三种类型:
\begin{enumerate}
  \item 主动式:\\ 
    照射目标的能源在导弹上,并且导弹接受
    目标反射回来的能量
  \item 半主动式:\\ 
    照射目标的能源在地面制导站或者其他位
    置,导弹上只有接收装置,制导距离比主动式
    大
  \item 被动式:\\ 
    目标本身就是辐射能源,不需要发射装置,弹上
    只有接受装置,导引头接受辐射能量
\end{enumerate}
\end{notice}
雷达自动寻的制导系统分类
\begin{equation*}
  \color{titleblue}
  \begin{cases}
    半主动式\text{:接收装置在导弹上}\\ 
    主动式\text{:发射和接收装置都在导弹上}
  \end{cases}
\end{equation*}
\begin{note}
缺点是容易受干扰
\end{note}
红外自动寻的制导系统分类
\begin{equation*}
  \color{titleblue}
  \begin{cases}
    半主动式\text{:接收装置在导弹上}\\ 
    主动式\text{:发射和接收装置都在导弹上}
  \end{cases}
\end{equation*}
\begin{equation*}
  \color{titleblue}
  \begin{cases}
    红外点源\text{:将目标看成是一个点}\\ 
    红外成像制导系统\text{:将目标的形状用红外绘制出来}
  \end{cases}
\end{equation*}
\begin{note}
红外导引头由红外探测系统和电子线路两部分组成.
\end{note}

激光自动寻的制导系统接收的是来自
激光发射器照射目标的反射激光,多用于半
主动式自动寻的制导系统中.

电视自动寻的制导系统的主要部件是
一部电视摄像机.
\begin{note}
电视自寻的制导系统具有被动式自动寻的制导
系统的优点,抗干扰性较强,隐蔽性好.
\end{note}
\section{舵机}
导弹控制系统的{\color{blue}执行机构}是根据控制指令信息来
驱动舵机带动操纵元件使弹体作出相应的姿态变
化.
\begin{equation*}
  \color{titleblue}
  舵机
  \begin{cases}
    气压式
    \begin{cases}
      冷气式舵机\text{:能源是贮存在容器中的
      压缩气体}\\ 
      燃气式舵机\text{:以燃气作为能源}\\ 
    \end{cases}\\ 
    液压式\text{:用一定压力的液压油作为舵机的能源}\\ 
    电磁式\\ 
    电动式\text{:主要是一台电机和减速装置}
  \end{cases}
\end{equation*}
\begin{notice}
几种舵机的优缺点
\begin{enumerate}
  \item 气压式舵机: 
  \begin{enumerate}
    \item 优点:\\ 
      简单,工作可靠
    \item 缺点:\\ 
      延时较大,快速性较差
  \end{enumerate}
\item 液压式舵机:
  \begin{enumerate}
    \item 优点:\\ 
      延时小,功率大,响应速度快
    \item 缺点:\\ 
      比其他舵机结构复杂,成本高
  \end{enumerate}
\item 电磁式舵机: 
  \begin{enumerate}
    \item 优点:\\ 
      结构简单,重量轻,需要的能量
      小,可靠性高
    \item 缺点:\\ 
      输出功率小
  \end{enumerate}
\end{enumerate}
\end{notice}

