% ! TEX root = ../mechanics.tex
\chapter{火箭导弹的动力装置}
\begin{equation*}
	\color{titleblue}
	发动机
	\begin{cases}
		火箭发动机
		\begin{cases}
			固体火箭发动机 \\
			液体火箭发动机
		\end{cases} \\
		空气喷气发动机
		\begin{cases}
			涡轮喷气发动机  \\
			涡轮风扇发动机  \\
			涡轮螺旋桨发动机 \\
			冲压发动机
		\end{cases}
	\end{cases}
\end{equation*}
{\color{blue}火箭动力装置}是为火箭飞行提供动力
的装置,推进系统保证火箭获得所需的
作战射程和飞行速度等特性.

发动机总体技术要求和依据如下:
\begin{enumerate}
	\item 功能和要求
	\item 发动机的总冲量,比冲量,推力
	      时间曲线
	\item 对发动机燃气流的特殊要求:无烟,少烟等
	\item 质量,尺寸要求
	\item 布局设计及结构对接要求
	\item 可靠性及寿命周期
	\item 安全性
	\item 启动特性
	\item 经济性
\end{enumerate}

选择发动机的一般原则
\begin{enumerate}
	\item 工作时间
	\item 对全弹质量的影响:发动机本身的质量,
	      迎面阻力,燃料消耗量
	\item 性能影响
	\item 对发动机的掌握程度
\end{enumerate}
选择方法
\begin{enumerate}
	\item 比冲与马赫数的关系
	\item 耗油量与马赫数的关系
	\item 推力质量比与马赫数的关系
	\item 推力最大迎风面积比与马赫数的关系
\end{enumerate}

火箭发动机的主要性能参数
\begin{equation*}
	\color{titleblue}
	\begin{cases}
		总冲量    \\
		推力     \\
		工作时间   \\
		火箭发动机的高度特性,空气喷气发动速度和
		高度特性   \\
		比冲     \\
		推力质量比  \\
		质量比    \\
		单位迎面推力 \\
		推力曲线设计及实现
	\end{cases}
\end{equation*}

{\bfseries 比冲}\index{比冲}:发动机消耗单位推进剂质量
产生的冲量.

{\bfseries 推力质量比}\index{推力质量比}:发动机的推力
与动力装置结构的净质量(不含燃料)的比值.

{\bfseries 质量比}\index{质量比}:推进剂质量与动力
装置总质量的比值.

{\bfseries 单位迎面推力}\index{单位迎面推力}:发动机
推力与其最大横截面积的比值.

\section{固体火箭发动机}
结构:{\color{blue}推进剂,燃烧室,喷管和点火装置}.

固体火箭发动机的特点:
\begin{itemize}
	\item 结构简单
	\item 可在短时间内产生大推力
	\item 使用简单,工作可靠,可长期贮存
	\item 机动性好
\end{itemize}

工作原理,主要分为两个过程:
\begin{enumerate}
	\item 装药在燃烧室里面燃烧的过程.装药
	      的大部分化学能释放出来,转变为燃烧产物的
	      热能和压力能
	\item 燃气在喷管中的膨胀过程.燃气随着压力
	      和温度的下降而膨胀,压力能和热能转变为
	      动能
\end{enumerate}

固体推进剂的分类:
\begin{equation*}
	\color{titleblue}
	\begin{cases}
		均质火药
		\begin{cases}
			单质药,硝化纤维为主 \\
			双基药
			\begin{cases}
				硝化纤维 \\
				硝化甘油
			\end{cases}
		\end{cases} \\
		异质火药
		\begin{cases}
			复合药
			\begin{cases}
				氧化剂 \\
				燃烧粘结剂
			\end{cases} \\
			黑火药
		\end{cases}
	\end{cases}
\end{equation*}

改进型双基药:双基药中加入过氯酸铵,铝粉或黑索金等.

对固体推进剂的要求
\begin{itemize}
	\item 能量性能,能量大,比重大
	\item 燃烧性能,燃烧时稳定性好
	\item 机械性能,贮存,运输时不会发生断裂
	\item 安定性能,具有良好的物理安定性和化学
	      安定性
	\item 生产经济性能
\end{itemize}

药柱截面的几何形状和燃烧表面积随时间的
变化规律决定了发动机推力随时间变化的
规律.

固体火箭发动机的分类
\begin{equation*}
  \color{titleblue}
  \begin{cases}
    自由装填式
    \begin{cases}
      发动机顶盖,头部支撑弹性件\\ 
      点火装置\\ 
      药柱,药柱包覆层,挡药板\\ 
      燃烧室壳体\\ 
      隔热层\\ 
      喷管
    \end{cases}\\ 
    浇注式
    \begin{cases}
      顶盖,底盖,堵盖\\ 
      点火装置\\ 
      燃烧室壳体\\ 
      石墨衬套\\ 
      药柱\\ 
      喷管
    \end{cases}
  \end{cases}
\end{equation*}

对于固体火箭发动机,采用石墨喉衬解决耐热问题.

\section{液体火箭发动机}
结构:{\color{blue}燃烧室,推进剂贮箱,输送系统}.

液体火箭发动机的燃料分类
\begin{enumerate}
	\item 单组元:\\
	      含燃烧机和氧化剂,特点是常温常压稳定,
	      加热,加压或接触触媒剂时分解.输送系统简单.
	      用于副系统中或涡轮泵组能源.
	\item 双组元:\\
	      燃烧剂和氧化剂在喷入燃烧室前不混合.
	\item 三组元:\\
	      液氧/烃+液氧三组元
\end{enumerate}

对液体推进剂的要求:
\begin{itemize}
	\item 高比冲
	\item 无腐蚀,无毒,物理性能稳定,冰点低,比重大,物理性能稳定
	\item 一种组元比热大,化学稳定性好,流量大,可用来冷却
	\item 粘度小
	\item 燃烧速度快,点火时间短,不发生有害的振荡
\end{itemize}

液体火箭发动机的组成
\begin{equation*}
	\color{titleblue}
	\begin{cases}
		推进剂输送系统  \\
		流量调节控制活门 \\
		冷却系统     \\
		推力室      \\
		固定零部件
	\end{cases}
\end{equation*}

推进剂输送系统的分类:
\begin{equation*}
	\color{titleblue}
	\begin{cases}
		挤压式输送系统 \\
		涡轮泵式输送系统
	\end{cases}
\end{equation*}

两种发动机的比较
\begin{enumerate}
	\item 固体火箭发动机的结构和设计比较简单,
	      液体火箭发动机的结构和设计比较复杂
	\item 固体火箭发动机的推力,工作时间受环境初温
	      影响比较大,而液体火箭发动机对环境
	      初温的敏感性小
	\item 液体火箭发动机可以随意开车和停车,
	      而固体火箭发动机不行
	\item 对于大推力,长时间工作的火箭导弹采用液体
	      火箭发动机比较轻.对于工作时间短的火箭导弹,
	      采用固体火箭发动机比较轻
	\item 液体火箭发动机的工作时间为50-1400秒,
	      而固体固体火箭发动机的工作时间最长才百余秒
	\item 液体推进剂的比冲要比固体推进剂的高
	\item 液体火箭发动机更容易实现推力调节,而固体火箭
	      发动机很难做到
	\item 液体火箭发动机的地面勤务处理比固体火箭麻烦
\end{enumerate}

固-液火箭发动机的组成
\begin{equation*}
	\color{titleblue}
	\begin{cases}
		发动机(包含固体药柱,喷管) \\
		液体推进剂贮箱        \\
		高压气瓶           \\
		活门             \\
		减压器
	\end{cases}
\end{equation*}
\section{空气喷气发动机}
空气喷气发动机的特点是本身只携带燃油,
氧化剂靠空气中的氧气.

涡轮喷气发动机可以分为
\begin{equation*}
	\color{titleblue}
	涡喷
	\begin{cases}
		轴流式涡轮喷气发动机\text{:}
		气流沿压气机轴的平行方向流动 \\
		离心式涡轮喷气发动机\text{:}
		靠离心力给空气增压
	\end{cases}
\end{equation*}

{\bfseries 原理}

空气由进气道进入发动机后,流速减小,压强增大.
经压气机进一步增大压强,在燃烧室与燃料掺混后,
燃烧,产生高温高压的燃气对涡轮做功,涡轮带动
压气机.燃气流过涡轮后,经喷管高速向后喷出,产生
推力.

吸气式喷气发动机的一些性能参数:

{\color{blue}推力,单位推力,推重比,单位耗油率,
单位迎风面积推力,噪声和排气污染}.

组成
\begin{equation*}
	\color{red}
	\begin{cases}
		进气道\text{:}整理进入发动机的气流,
		消除紊乱的涡流
		\\
		压气机\text{:}压气机通道做成收敛形状,
		发动机匣上装有静子叶片,压气机轴上有
		转子叶片
		\\
		燃烧室\text{:}一部分空气与燃油混合,雾
		化,燃烧,变成高温高压的燃气
		\\
		涡轮\text{:}为压气机提供能量
		\\
		加力燃烧室\text{:}利用燃气中剩余的氧气
		,再次组织燃烧,提高燃气的温度
		\\
		喷管\text{:}飞行速度不太高时,采用收敛
		喷管,飞行速度较高时,采用拉瓦尔喷管
	\end{cases}
\end{equation*}

涡轮喷气发动机的特点
\begin{enumerate}
	\item 必须依靠外界能源启动
	\item 飞行高度增加,空气密度下降,推力下降
	\item 构造复杂,重量大
	\item 主要用在飞机上
\end{enumerate}

由于涡轮喷气发动机喷出的气流温度还很高,
尚且有相当一部分能量没有利用,因此提出
涡轮风扇发动机.

涡扇在涡喷的基础上,将气流分成两股,一股
参与燃烧,另外一股不参与燃烧直接喷出(或与
燃气混合后喷出).并且在发动机前缘加装一个
风扇,用来提高进气量.采用两组涡轮带动
风扇和压气机.

和涡轮喷气发动机相比优点是:
\begin{itemize}
	\item 耗油率小
	\item 推力大
\end{itemize}

\begin{note}
	涡扇发动机的推力大部分都是由外涵道的气流提供的,内涵
	道气流的能量大部分用来做功了
\end{note}

可以在外涵道上再次喷油燃烧做成一个加力燃烧室.
涡扇发动机的迎风面积比涡喷大些.

\section{冲压发动机}
冲压发动机的原理同涡喷发动机,同样具有
三个基本过程:{\color{blue}压缩过程,燃烧
过程,膨胀过程}.
区别是冲压发动机没有了压气机,靠速度冲压
将空气压缩.

冲压发动机的组成及功能
\begin{enumerate}
  \item 进气道:\\ 
    引入空气,实现压缩过程,提高气流压力.依靠
    高速气流的滞止过程进行压缩.理想情况下
    增压比很高
  \item 燃烧室:\\ 
    实现燃烧的地方,装有预燃室,点火器,燃油喷嘴.
  \item 尾喷管:\\ 
    高温高压气流实现膨胀加速.
  \item 燃油供给及自动调节系统:\\ 
    感受外界气流参数,调节燃油,保证正常燃烧.
\end{enumerate}

优点:
\begin{itemize}
  \item 构造简单,重量低,成本低
  \item 高速飞行状态下,经济性好,耗油率低
\end{itemize}

缺点:
\begin{itemize}
  \item 低速时推力小,耗油率高,静止时不能
    产生推力
  \item 冲压发动机对飞行状况的变化很敏感
  \item 随推力增加,发动机的体积和直径都越来越大
\end{itemize}

\section{火箭冲压发动机}
\begin{enumerate}
  \item 固体火箭冲压发动机:\\ 
    由进气道,燃气发生器,引射掺混补燃室组成.助推器是
    一个典型的固体火箭发动机.
  \item 固体燃料冲压发动机:\\ 
    助推器药柱与冲压发动机共用一个燃烧室
  \item 液体燃料冲压发动机:\\ 
    助推器和液体燃料发动机共用一个燃烧室
\end{enumerate}

