% ! TEX root = ../mechanics.tex
\chapter{引信和战斗部}
\section{引信}
引信的三大功能
\begin{enumerate}
  \item 在引信生产装配,运输,贮存,装填,发射
    以及发射后的弹道起始段不能提前作用
  \item 感受目标信息并加以处理,确定战斗部的最佳
    起爆位置
  \item 向战斗部输出足够的起爆信息,完全地引爆
    战斗部
\end{enumerate}
引信的组成
\begin{equation*}
  \color{titleblue}
\begin{cases}
  安全系统\\ 
  发火控制系统\\ 
  传爆系列
\end{cases}
\end{equation*}
火工元件的发火方式
\begin{equation*}
  \color{titleblue}
  发火方式
  \begin{cases}
    机械发火
    \begin{cases}
      针刺发火\\ 
      撞击发火\\ 
      绝热压缩发火
    \end{cases}\\ 
    电发火\\ 
    化学发火
  \end{cases}
\end{equation*}

引信的分类
\begin{equation*}
  \color{titleblue}
  \begin{cases}
    触发引信
    \begin{cases}
      按时间分
      \begin{cases}
        瞬发引信\\ 
        延期引信
      \end{cases}\\ 
      按触发力源分
      \begin{cases}
        触发引信\\ 
        惯性引信
      \end{cases}\\ 
      按起爆能源分
      \begin{cases}
        机械触发引信\\ 
        压电引信\\ 
        电触发引信
      \end{cases}
    \end{cases}\\ 
    非触发引信
    \begin{cases}
      按受激励特征不同分
      \begin{cases}
        光学引信\\ 
        无线电引信\\ 
        气压引信
      \end{cases}\\ 
      按控制时间分
      \begin{cases}
        火药时间引信\\ 
        钟表时间引信\\ 
        电力计时引信
      \end{cases}
    \end{cases}
  \end{cases}
\end{equation*}

对火箭引信的特殊要求
\begin{enumerate}
  \item 引信应保证在低过载条件下平时安全
    与发射时可靠解除保险
  \item 引信要有足够的解除保险距离
  \item 火箭发动机工作不正常时,引信应保证
    不解除保险
  \item 引信应在大着角碰击目标时作用可靠
  \item 引信应是隔离雷管的
\end{enumerate}

\section{战斗部}
战斗部由{\color{blue}装填物,壳体,引信,传爆系列}组成.

{\bfseries 装填物}\index{装填物}是破坏目标的能源和工质.
主要有炸药和核装料.

{\bfseries 壳体}\index{壳体}是装载装填物的容器,同时也是
战斗部连接其他零部件的基体.还可以产生破片.

{\bfseries 引信}\index{引信}是适时引爆战斗部的引爆装置.

{\bfseries 传爆系列}\index{传爆系列}是一种能量放大器.
把目标给予的起始能量转变为爆炸波或火焰.

战斗部分类
\begin{equation*}
  \color{red}
  \begin{cases}
    常规战斗部
    \begin{cases}
      爆破战斗部\\ 
      聚能破甲战斗部\\ 
      杀伤战斗部
      \begin{cases}
        无控破片杀伤战斗部\\ 
        可控破片杀伤战斗部\\ 
        连续杆杀伤战斗部\\ 
        多聚能杀伤战斗部
      \end{cases}\\ 
      综合作用战斗部\\ 
      碎甲战斗部
    \end{cases}\\ 
    核战斗部
    \begin{cases}
      原子弹头\\ 
      氢弹头\\ 
      中子弹头
    \end{cases}\\ 
    特种战斗部
    \begin{cases}
      激光战斗部\\ 
      X射线战斗部\\ 
      化学毒剂战斗部\\ 
      燃烧战斗部\\ 
      发烟或发光战斗部
    \end{cases}
  \end{cases}
\end{equation*}

炸药是一种爆炸物质,具有一下特征:
\begin{itemize}
  \item 爆炸时产生气体
  \item 爆炸时释放热量
  \item 爆炸速度极快
\end{itemize}
起爆炸药的外能主要有{\color{blue}机械能起爆,
电能起爆,爆炸能起爆,热能起爆}.

炸药的爆炸性能主要有{\color{blue}感度,威力,烈度}.

杀伤战斗部的结构形式:{\color{blue}破片式
结构,条状式结构,聚能效应结构}.

几种典型的破片杀伤式战斗部
\begin{equation*}
  \color{titleblue}
  \begin{cases}
    壳体刻槽式杀伤战斗部\\ 
    装药表面刻槽式杀伤战斗部\\ 
    圆环叠加点焊式杀伤战斗部\\ 
    预制破片式杀伤战斗部
  \end{cases}
\end{equation*}

\chapter{其他知识}
弹道式导弹的特点:
\begin{enumerate}
  \item 弹道式导弹的弹道分为主动段
    和被动段
  \item 弹道式导弹的弹道大部分是在大气层
    之外,一般采用火箭发动机
  \item 弹道式导弹采用燃气舵,空气舵面,可
    偏摆的发动机来操作导弹转弯
  \item 弹道式导弹的飞行速度相当大
  \item 存在再入大气层问题
\end{enumerate}
