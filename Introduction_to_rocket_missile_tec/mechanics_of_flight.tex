% ! TEX root = ../mechanics.tex
\chapter{火箭导弹的飞行力学}
作用在火箭弹的力有{\color{blue}发动机
推力$P$,火箭弹重力$G$,总空气动力$R$.}作
用在火箭弹上主要力矩有{\color{blue}俯仰力
矩$M_\theta$,偏航力矩$M_\Phi$,滚转
力矩$M_\gamma$}.
\begin{note}
	总空气动力作用在压心上,同时产生空气力矩.
\end{note}
火箭弹的运动方程中共有$9$个参数,即
速度$v$,弹道倾角$\theta$,攻角$\alpha$,俯仰
角$\Phi$,弹道坐标参数$x$,$y$和时间$t$,共有
$8$个方程,除时间$t$外,其余参数都可以看成是
时间$t$的函数.
\section{火箭弹的弹道}
\subsection{弹道特性}
火箭弹的弹道具有两个连续的区段,即
{\color{blue}主动段和被动段}.

若火箭弹在真空中飞行,无空气动力作用,射程
$x$则为
\[
	x=\frac{v_0^2 \sin 2\theta_0}{g }
\]
\begin{note}
	射程仅仅取决于主动段终点的速度$v_0$和
	弹道倾角$\theta_0$
\end{note}
\section{密集度问题}
设计精度用对目标的命中概率描述,包含
{\color{blue}准确度和密集度}两部分.
{\bfseries 准确度}\index{准确度}是
指火箭弹炸点散布中心偏离射击指向点的程度.
{\bfseries 密集度}\index{密集度}是
各火箭弹炸点围绕散布中心的密集程度.
\begin{note}
	密集度高不代表准确度好;准确度好不代表
	密集度高
\end{note}
引起导弹散布的误差根源:
\begin{equation*}
	\color{red}
	\begin{cases}
		方向密集度
		\begin{cases}
			横风的不一致和扰动气流 \\
			火箭弹离轨时的初始扰动 \\
			推力偏心和空气动力偏心 \\
			质量偏心和动不平衡
		\end{cases} \\
		距离密集度
		\begin{cases}
			火药装药质量不一致        \\
			火药装药比冲量不一致       \\
			火箭弹弹体质量不一致       \\
			火箭弹弹体加工公差引起的比冲量变化 \\
			飞行过程中阻力变化
		\end{cases}
	\end{cases}
\end{equation*}
\section{导弹的飞行弹道}
制导弹道按形成的特点可以分为:
\begin{equation*}
	\color{titleblue}
	\begin{cases}
		用于攻击固定目标的\text{:}整个
		弹道或大部分弹道在发射前就被制导
		系统预先确定 \\
		用于攻击活动目标的\text{:}弹道
		在发射前不能预先确定,是一种随机弹道
	\end{cases}
\end{equation*}
\begin{notice}
	弹道式导弹和无控火箭弹弹道的区别是:
	\begin{enumerate}
		\item 主动段有控制力和控制力矩
		\item 被动段不是抛物线形式,而是
		      椭圆曲线形弹道
	\end{enumerate}
\end{notice}
\subsection{弹道式导弹弹道的分段}
\begin{equation*}
	\color{titleblue}
	\begin{cases}
		主动段
		\begin{cases}
			垂直上升段  \\
			转弯飞行段  \\
			发动机关车段 \\
		\end{cases} \\
		被动段
		\begin{cases}
			自由飞行段 \\
			再入段
		\end{cases}
	\end{cases}
\end{equation*}
\begin{note}
	弹道式导弹只在主动段上制导
\end{note}
\section{有翼式导弹的弹道}
有翼式导弹的飞行路线是根据某种
与目标的相对关系来控制的.

导弹和目标之间的相对运动所
遵循的规律称为{\color{blue}导引规律},此时
的弹道称为{\color{blue}导引弹道}.

导弹与目标的连线称为
{\bfseries 目标线}\index{目标线}.

导弹速度矢量与目标线的夹角称为
{\bfseries 导弹前置角}\index{导弹前置角}.

飞行弹道可分为
\begin{equation*}
	\color{titleblue}
	\begin{cases}
		追踪导引弹道\text{:}导弹的速度矢量
		始终指向目标         \\
		平行接近法的飞行弹道\text{:}导引过程
		中目标线在空间的位置始终不变 \\
		三点法的飞行弹道\text{:}导弹,目标,制导站
		始终连成一线
	\end{cases}
\end{equation*}
\begin{notice}
	几种弹道的优缺点:
	\begin{enumerate}
		\item 追踪导引弹道:
		      \begin{enumerate}
			      \item 优点:\\
			            简单
			      \item 缺点:\\
			            当导弹需要迎面射击目标时,或追踪
			            近距离高速目标时,可能会出现弯曲度
			            过大的弹道,导弹要承受过大的载荷
		      \end{enumerate}
		\item 平行接近法弹道:
		      \begin{enumerate}
			      \item 优点:\\
			            当导弹迎向目标或近距离高速飞行目标时
			            ,弹道弯曲程度小,法向过载要求小
			      \item 缺点:\\
			            制导系统复杂,实现困难
		      \end{enumerate}
		\item 三点法的飞行弹道:
		      \begin{enumerate}
			      \item 优点:\\
			            制导简单,抗电子干扰能力强
			      \item 缺点:\\ 
              当导弹迎着目标或近距离高速飞行目标时,
              弹道弯曲程度大,法向过载要求大
		      \end{enumerate}
	\end{enumerate}
\end{notice}
\section{导弹的机动性,过载,稳定性及操纵性}
{\bfseries 机动性}\index{机动性}:指导弹能
迅速改变飞行速度的大小和方向的能力.
\begin{note}
可用导弹在飞行过程中所能产生的切向加速度和
法向加速度的大小来评定导弹机动性的好坏
\end{note}

对于有翼导弹,法向机动性取决于法向空气动力的
大小;对于采用推力控制飞行的导弹,法向机动性
取决于发动机推力的大小及其可能偏向弹体轴线
角度的大小.

\begin{notice}
影响导弹机动性的因素
\begin{enumerate}
  \item 导弹的气动特性
  \item 导弹的质量大小
  \item 弹道倾角$\theta$的大小
  \item 飞行的高度.高度越高大气密度越低,
    法向机动性就要下降.
  \item 弹翼的面积.弹翼面积越大,法向机动性
    就越好.
\end{enumerate}
\end{notice}

{\bfseries 过载}\index{过载}指作用在导弹
上外力的大小和导弹重量的比值.
\[
  n=\frac{N}{G }
\]
\begin{note}
过载是一个向量,方向与外力$\mathbf{N}$的
方向一致.是一个无量纲的量.
\end{note}
\begin{notice}
限制导弹过载的因素:
\begin{enumerate}
  \item 导弹操纵元件的偏转范围是有限的
  \item 产生气动法向力的攻角和侧滑角不可能
    太大,不能超过它们的临界值
  \item 导弹弹体的结构不允许法向力很大,否则
    弹体将遭到破坏
\end{enumerate}
\end{notice}

{\bfseries 稳定性}\index{稳定性}指导弹在飞行
过载中,由于受到某种干扰,使其偏离原来的飞行状
态,当干扰消失后,导弹恢复原来飞行状态的能力.
\begin{note}
只要保证气动力的焦点位于质心之后,并且有一段距离,
就可以保证攻角$\alpha$是稳定的,如果弹上装有自稳定
系统,则无此要求
\end{note}

{\bfseries 操纵性}\index{操纵性}指导弹在操纵元件
发生动作时,改变其原来飞行状态的能力以及对此
反应快慢的程度.
\begin{note}
导弹的稳定性和操纵性时对立统一的
\end{note}
\begin{notice}
稳定性和操纵性的对立统一:
\begin{enumerate}
  \item 对立性:\\ 
    导弹的操纵性越好,导弹就越容易改变其原来的
    飞行状态;导弹的操纵性越差,导弹就越难改变
    其原来的飞行状态.因此,提高导弹的操纵性,就
    会削弱导弹的稳定性,反之亦然.
  \item 统一性:\\ 
    静稳定性差或者静不稳定的导弹,则要求自动稳定
    系统使操纵元件发生动作从而产生操纵力矩,以便
    对导弹进行操纵,来克服外加干扰维持导弹的稳定.
    在这种情况下,如果导弹的操纵性好,导弹在自动
    稳定系统的作用下,能够较快地改变其飞行状态,
    使导弹快速达到稳定.因此,导弹的操纵性有助于
    加强导弹的稳定性.
\end{enumerate}
\end{notice}


