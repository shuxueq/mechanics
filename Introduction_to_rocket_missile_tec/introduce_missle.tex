% ! TEX root = ../mechanics.tex
%火箭导弹概论

\part{火箭导弹技术引论}
\chapter{概述}
\section{火箭弹的特点}
相较于身管武器(火炮),火箭弹武器系统的特点是:
\begin{enumerate}
	\item 有较高的飞行速度
	\item 发射时无后座力
	\item 发射时的过载系数小
\end{enumerate}
缺点是:
\begin{enumerate}
	\item 密集度较差
	\item 容易暴露发射阵地
	\item 造价比相同威力的炮弹高
\end{enumerate}

\section{火箭导弹的分类}
按照稳定方式分类的火箭弹
\begin{equation*}
	\color{titleblue}
	\begin{cases}
		 \text{尾翼稳定火箭弹} \\
		 \text{旋转稳定火箭弹}
	\end{cases}
\end{equation*}
\begin{notice}
	\begin{itemize}
		\item 尾翼稳定火箭弹的特点:\\
		      弹身较长,装药量大,射程远,
		      抗干扰能力高,稳定性高.
		\item 旋转稳定火箭弹的特点:\\
		      火箭弹高速旋转,能减少推力偏心和
          质量偏心的不良影响,提高
		      密集度;弹身较短,没有尾翼,
          易于实现机械化装填但弹长
          受限,难以增加发动机装药量.
	\end{itemize}
\end{notice}


按照用途方式分类的火箭弹
\begin{equation*}
	\color{titleblue}
	\begin{cases}
		 \text{反坦克火箭弹} \\
		 \text{野战火箭弹}  \\
		 \text{航空火箭弹}
	\end{cases}
\end{equation*}

\section{火箭弹的战术技术要求}
\begin{equation*}
	\color{titleblue}
	\begin{cases}
		 \text{射程}    \\
		 \text{威力}    \\
		 \text{密集度}   \\
		 \text{机动性}   \\
		 \text{安全可靠性}
	\end{cases}
\end{equation*}

\section{导弹}
按照射程可以分为
\begin{equation*}
	\color{titleblue}
	\begin{cases}
		 \text{近程导弹} \\
		 \text{中程导弹} \\
		 \text{远程导弹} \\
		 \text{洲际导弹}
	\end{cases}
\end{equation*}

\begin{equation*}
	\color{titleblue}
	\begin{cases}
		 \text{面对面导弹}   \\
		 \text{面对空导弹}   \\
		 \text{空对面导弹}   \\
		 \text{空对空导弹}   \\
		 \text{反舰(潜)导弹} \\
		 \text{反坦克导弹}
	\end{cases}
\end{equation*}

\subsection{组成部分}
\begin{equation*}
	\color{titleblue}
	\begin{cases}
		 \text{动力装置:为导弹提供飞行动力}       \\
		 \text{制导系统:将导弹导向目标}         \\
		 \text{战斗部:直接毁伤目标,完成战斗任务的部分} \\
		 \text{弹体:以安装战斗部,控制系统,动力装置,
		推进剂和弹上电源等}                     \\
		 \text{给弹上各分系统工作用电的电能装置}
	\end{cases}
\end{equation*}

\subsection{导弹战斗技术指标}
\begin{equation*}
	\color{titleblue}
	\begin{cases}
		 \text{战术要求}
		\begin{cases}
			 \text{导弹类别}  \\
			 \text{战斗部威力} \\
			 \text{命中概率}  \\
			 \text{飞行性能}
		\end{cases}     \\
		 \text{技术要求}
		\begin{cases}
			 \text{发动机类别}     \\
			 \text{导弹极限尺寸和重量} \\
			 \text{所用材料限制}
		\end{cases} \\
		 \text{使用维护要求}
		\begin{cases}
			 \text{部件互换性}     \\
			 \text{运输方便}      \\
			 \text{操作安全,贮存期限}
		\end{cases}
	\end{cases}
\end{equation*}

